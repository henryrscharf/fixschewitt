\documentclass[12pt]{article}
\usepackage{hyperref} % for hyperlink
\usepackage{graphicx} % for figures
\usepackage{caption}
% \usepackage{subcaption}
\usepackage{enumerate}
\usepackage{amsmath,amsfonts,amssymb} % for mathematical symbols
\usepackage{mathtools}
\usepackage[margin=1in]{geometry}
\usepackage{float}

\floatstyle{ruled}
\newfloat{code}{thp}{lop}
\floatname{code}{Code Segment}

%\setlength{\topmargin}{-0in}
%\setlength{\textheight}{9in}
%\setlength{\evensidemargin}{-.25in}
%\setlength{\oddsidemargin}{-.25in}
%\setlength{\textwidth}{6.5in}                   % set up margin

\setcounter{page}{1}
\pagenumbering{arabic}

% \graphicspath{}
% \DeclareGraphicsExtensions{.pdf}

\begin{document}

\begin{center}
{\bf \large Tutorial on Parallel Programming in R}\\
{ \large Josh Hewitt \& Henry Scharf}\\
\today{}
\end{center}

\vspace{.2in}

\section{Description}

As the size of data increases at a rapid pace, the number of
individual computations for even standard analyes is growing at a
staggering rate. The rate at which individual processors can perform
these compuatations is no longer keeping pace with the volume of data,
and so several useful techniques can become prohibitively slow to
implement. Luckily, the cost of processors has continued to decline
rapidly, and so multi-core systems are common place even in personal
computers. Many of these slow techniques involve several independent
tasks which may be spread over several cores, thereby significantly
reducing the total computation time. In this tutorial, we will focus
on identifying these situations, and using a few packages in R to
`parallelize' sequential code. We illustrate the process with several
commonplace examples which will include some of the following:

\begin{enumerate}[(a)]
\item bootstrapping (CI for linear regression coef)
\item cross validation ?
\item simulation?
\item sensitivity analysis in bayesian statistics?
\item large datasets? (eg: genetic?)
\item permutation tests?
\end{enumerate}

\textsc{R packages used}
\begin{enumerate}[(a)]
\item `foreach'
\item `multicore'
\item `Rhadoop'
\end{enumerate}

\section{Outline and Objectives}
\begin{enumerate}
  \item Identify parallelizable computation tasks
  \item Extend a working knowledge of R programming to take advantage
    of multiple cores
  \item 
  \item 
\end{enumerate}

\section{About the Instructor}

Henry Scharf is a PhD student at Colorado State University with a
strong background in both teaching and computational statistics. He
received his Masters in Education from the University of Arizona, and
is an instructor at CSU. He has worked in conjunction with the 
National Renewable Energy Labratory on questions surrounding
prioritized compression of massive datasets sensistive to specific
secondary analysis.

\section{Relevance to Conference Goals}


Theme 3: Model validation and comparison approaches.

Theme 4: Software and programming methods to obtain, clean, describe,
or analyze data.


% \begin{table}[ht]
%   \caption{}
%   \label{}
%   \begin{tabular}{|l|cc|}
%     \hline
%     \hline
%     \hline
%   \end{tabular}
% \end{table}


% \begin{figure}[h!]
%   \centering
%   \includegraphics[width=\linewidth]{}
%   \caption{}
%   \label{fig:}
% \end{figure}

% \appendix

% \begin{thebibliography}{9}
% \bibitem{Scharf}

% \end{thebibliography}

\end{document}